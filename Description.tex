Instances in a bag, $\mathbf{x}_i$, are observations from an underlying distribution. 
In the collective assumption, the instances of two bags belonging to the same class are (implicitly) assumed to be drawn from the same distribution.
We will generalise this assumption by a hierarchical approach: 
\begin{align}
  \mathbf{x}_i & \sim P(X|\mathbf{\Theta}_{b_k}) \\
  \mathbf{\Theta} & \sim P(\mathbf{\Theta}|\mathbf{\tau}_j), 
\end{align}
where $\mathbf{\tau}$ is a fixed parameter vector defined by the class. 

This could be a simple embedded approach, saying $D(\mathbf{\theta}, \mathbf{\tau})$, but instead the observations are used directly. 

How does the hierarchical assumption describe real-life bags?
If we go back to the tumour image example, we can assume that tumour pixels are drawn from the distribution $f(\mathbf{x}|\theta_{tum})$ and normal tissue pixels are drawn from the distribution $f(\mathbf{x}|\theta_{norm})$. Then, a pixel from an image containing a tumour will be drawn from the distribution 
\begin{align}
  f_{pix}(\mathbf{x}|\theta_{tum},\theta_{norm}, \pi_{tum}) = \pi_{tum}f(\mathbf{x}|\theta_{tum}) + (1-\pi_{tum})f(\mathbf{x}|, \theta_{norm})
\end{align}
where $0 < \pi_{tum} \leq 1$ is the proportion of the image occupied by the tumour.  
A pixel from an image not containing a tumour will be drawn from the same distribution, but with $\pi_{tum} = 0$.
In this simple example, the $\theta$'s are kept constant, but we can imagine without difficulty situations where the $\theta$'s will vary according to some distribution. 
